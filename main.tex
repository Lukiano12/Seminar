\documentclass[a4paper, 12pt]{article} % Artikel-Klasse

%---------------------------------------------------------
% Encoding, language, quotes
%---------------------------------------------------------
\usepackage[utf8]{inputenc}
\usepackage[ngerman]{babel}       % Deutsche Sprache und Silbentrennung
\usepackage{csquotes}             % Für korrekte Anführungszeichen
\usepackage{listings}
\usepackage{xcolor}

%---------------------------------------------------------
% Graphics & PDF
%---------------------------------------------------------
\usepackage{graphicx}
\usepackage{pdfpages}             % Einbinden von PDF-Seiten
\usepackage{caption}              % Verbesserte Bildunterschriften
\usepackage{subcaption}
\usepackage{comment}


% Customize settings
% Customize settings for a more compact style
\lstset{
    language=Java,                 % Specify Java language for syntax highlighting
    basicstyle=\ttfamily\small,    % Use smaller monospaced font for code
    keywordstyle=\color{blue},     % Style for keywords
    commentstyle=\color{gray},     % Style for comments
    stringstyle=\color{red},       % Style for strings
    numbers=none,                  % No line numbers
    breaklines=true,               % Break long lines
    frame=none,                    % No frame around the code
    xleftmargin=0pt,               % Remove left margin
    xrightmargin=0pt,              % Remove right margin
    aboveskip=5pt,                 % Reduce space above the code block
    belowskip=5pt                  % Reduce space below the code block
}

%---------------------------------------------------------
% Math, units, spacing, etc.
%---------------------------------------------------------
\usepackage{siunitx}
\usepackage{setspace}
\usepackage{textgreek}

% Add float package for "H" float option
\usepackage{float}

%---------------------------------------------------------
% Other packages
%---------------------------------------------------------
\usepackage{ifthen}
\usepackage{acronym}
\PassOptionsToPackage{hyphens}{url} % URLs in Hyperlinks umbrechen
\usepackage[breaklinks=true]{hyperref} 
\usepackage{array}                % Bessere Tabellenformatierung
\usepackage{enumitem}             % Kontrolle über Listen-Layouts
\usepackage{nomencl}
\usepackage{scrlayer-scrpage}     % Header und Footer

% Adjust header and footer heights
\setlength{\headheight}{14.5pt}
\setlength{\footheight}{34.16666pt}

%---------------------------------------------------------
% Bibliography (biblatex mit Biber)
%---------------------------------------------------------
\usepackage[backend=biber, style=numeric]{biblatex}  

\addbibresource{literatur.bib}  

%---------------------------------------------------------
% Platzhalter
%---------------------------------------------------------
\newcommand{\titel}{Muster-Klimaprozess mit Shell in den Haag, weitere Beispiele für Muster-Klimaprozesse}
\newcommand{\untertitel}{}
\newcommand{\arbeit}{Seminar}
\newcommand{\studiengang}{Elektrotechnik}
\newcommand{\studienrichtung}{Fahrzeugelektronik}
\newcommand{\autor}{Luka Tadic}
\newcommand{\abgabe}{}
\newcommand{\bearbeitungszeitraum}{}
\newcommand{\matrikelnr}{5726700}
\newcommand{\kurs}{TFE22–1}
\newcommand{\firma}{}
\newcommand{\betreuerfirma}{}
\newcommand{\gutachterdhbw}{}
\newcommand{\jahr}{2025}

%---------------------------------------------------------
% Header und Footer mit Linien
%---------------------------------------------------------
\clearpairofpagestyles{}         % Standard-Stile löschen

% Header with consistent logo placement and line position
\ohead{%
    \raisebox{1.5cm}[0pt][0pt]{% Raise the logo well above the line
        \includegraphics[width=3cm]{images/DHBW_d_R_FN_46mm_4c}%
    }%
    \\[-1.5cm] % Move the header line down
    \rule{\textwidth}{0.4pt} % Horizontal rule for the header line
}


% Footer with consistent alignment and contents below the line
\setkomafont{pagefoot}{\normalfont} % Ensure consistent font style
\cfoot{%
    \rule{\textwidth}{0.4pt}\\ % Horizontal rule
    \vspace{0.3em} % Small vertical space
    \begin{tabular}{@{}p{0.33\textwidth}p{0.33\textwidth}p{0.33\textwidth}@{}}
        \arbeit~& \centering \autor~& \raggedleft~\thepage%
    \end{tabular}
}


\pagestyle{scrheadings}        % Stil aktivieren

%---------------------------------------------------------
% Dokumentbeginn
%---------------------------------------------------------
\begin{document}
\sloppy

%---------------------------------------------------------
% Inhaltsverzeichnis
%---------------------------------------------------------
\tableofcontents

\clearpage



\section{Einleitung Klimaprozesse – Definition und Bedeutung}
\subsection{Was ist ein Klimaprozess?}
\subsection{Relevanz von Klimaprozessen für Umwelt und Gesellschaft}
\section{Der Shell-Klimaprozess in den Haag}
\subsection{Prozess und Ergebnisse}
\subsection{Bedeutung und Auswirkungen}
\section{Weitere wichtige Klimaprozesse}
\subsection{Saúl Luciano Lliuya gegen RWE}
\subsection{Der Shell-Nigeria-Prozess}
\subsection{Klimaklage der Stiftung Urgenda gegen den niederländischen Staat}
\section{Fazit}
\section{Quellen}

\clearpage

\section{Stichpunkte Quellen}

\textbf{\cite{8JahreKlimaklage} 8 Jahre Klimaklage gegen RWE}

Der Artikel \glqq8 Jahre Klimaklage gegen RWE\grqq~auf der Website rwe.climatecase.org 
fasst die Entwicklungen der seit 2015 laufenden Klage des peruanischen Bergführers und Kleinbauern 
Saúl Luciano Lliuya gegen den deutschen Energiekonzern RWE zusammen. 

 \begin{itemize}
    \item Klageeinreichung: Am 24. November 2015 reichte Saúl Luciano Lliuya eine zivilrechtliche Klage gegen RWE beim Landgericht Essen ein.
    \item Klagegrund: Der Kläger macht RWE für die durch den Klimawandel verursachte Gletscherschmelze verantwortlich, die den Palcacocha-See oberhalb von Huaraz gefährlich anschwellen lässt und somit eine potenzielle Flutgefahr für die Stadt darstellt.
    \item Forderung: Saúl Luciano Lliuya verlangt, dass RWE sich entsprechend seines Anteils von 0,47 \% an den globalen CO$_2$-Emissionen finanziell an den notwendigen Schutzmaßnahmen beteiligt.
    \item Verfahrensstand: Nach mehreren Jahren juristischer Auseinandersetzungen steht nun ein entscheidender Verfahrensschritt an: Eine mündliche Verhandlung ist für März 2024 geplant.

 \end{itemize}

 Dieser Fall gilt als bedeutender Präzedenzfall für die Frage, ob große Emittenten für den Schutz vor Klimarisiken haftbar gemacht werden 
 können.

 \textbf{\cite{agreement2015paris} Paris Agreement}

 \begin{itemize}
    \item Ziel: Die globale Erwärmung auf deutlich unter 2 °C gegenüber dem vorindustriellen Niveau zu begrenzen, mit dem Bestreben, 1,5 °C nicht zu überschreiten.
    \item Verpflichtungen: Alle Vertragsstaaten müssen nationale Klimaschutzpläne (NDCs – Nationally Determined Contributions) vorlegen und regelmäßig aktualisieren.
    \item Klimaneutralität: Die Weltwirtschaft soll in der zweiten Hälfte des 21. Jahrhunderts klimaneutral werden.
    \item Finanzierung: Industrieländer sollen 100 Milliarden US-Dollar pro Jahr für Klimaschutzmaßnahmen in Entwicklungsländern bereitstellen.
    \item Transparenz: Einführung eines Mechanismus zur regelmäßigen Überprüfung der Fortschritte.
 \end{itemize}

 Das Abkommen trat 2016 in Kraft und ist völkerrechtlich verbindlich, jedoch ohne direkte Sanktionsmechanismen.

 \textbf{\cite{ArmandoFerraoCarvalho} Armando {{Ferrão Carvalho}} and {{Others}} v. {{The European Parliament}} and the {{Council}}}

 Der Fall Armando Ferrão Carvalho u. a. gegen das Europäische Parlament und den Rat (auch bekannt als \glqq{}People's Climate Case\grqq{}) war eine
  Klage von zehn Familien aus verschiedenen EU-Mitgliedstaaten sowie aus Kenia und Fidschi, die in der Landwirtschaft und im Tourismus tätig sind.
   Sie reichten die Klage beim Gericht der Europäischen 
 Union ein und forderten strengere Treibhausgasemissionsziele der EU, da die bestehenden Maßnahmen ihre Grundrechte beeinträchtigen würden.

 \begin{itemize}
    \item Angefochtene Rechtsakte: Die Kläger wandten sich gegen drei EU-Rechtsakte:
    \item Die Emissionshandelsrichtlinie (Richtlinie 2003/87/EG), die Emissionen großer Energieerzeugungsanlagen regelt.
    \item Die Lastenverteilungsverordnung (Verordnung (EU) 2018/842), die Emissionen aus Sektoren wie Industrie, Verkehr, Gebäude und Landwirtschaft betrifft.
    \item Die LULUCF-Verordnung (Verordnung (EU) 2018/841), die Emissionen und Entnahmen im Zusammenhang mit Landnutzung, Landnutzungsänderungen und Forstwirtschaft regelt.
    \item Argumentation der Kläger: Sie argumentierten, dass die unzureichenden Emissionsminderungen höhere Rechtsnormen verletzen, die grundlegende Rechte wie Gesundheit, Bildung, Beruf und Gleichbehandlung schützen, sowie Verpflichtungen zum Umweltschutz.
    \item Forderung: Die Kläger verlangten, dass die EU ihre Treibhausgasemissionen bis 2030 um 50–60 \% gegenüber dem Niveau von 1990 reduziert, um ihren rechtlichen Verpflichtungen nachzukommen.
 \end{itemize}

 \textbf{Verfahrensverlauf:}
 \begin{itemize}
    \item Erste Instanz: Das Gericht der Europäischen Union wies die Klage als unzulässig ab, da die Kläger keine unmittelbare und individuelle Betroffenheit nachweisen konnten.
    \item Berufung: Die Kläger legten Berufung beim Gerichtshof der Europäischen Union ein, der die Entscheidung der Vorinstanz bestätigte und die Berufung am 25. März 2021 zurückwies.
 \end{itemize}

 Dieser Fall verdeutlicht die Herausforderungen bei Klimaklagen auf EU-Ebene, insbesondere hinsichtlich der Anforderungen an die Klagebefugnis von Einzelpersonen.

 \textbf{\cite{bandaLitigatingClimateChange2017} Litigating {{Climate Change}} in {{National Courts}}: {{Recent Trends}} and {{Developments}} in {{Global Climate Law}}}
 \begin{itemize}
    \item Überblick über Klimaklagen: Der Artikel untersucht, wie nationale Gerichte weltweit zunehmend in Klimaklagen eingebunden werden und welche Rolle sie in der globalen Klimapolitik spielen.
    \item Fallanalysen und Trends: Es werden jüngste Fallbeispiele vorgestellt, die den Einsatz unterschiedlicher juristischer Strategien und Argumente in Klimaschutzklagen beleuchten.
    \item Juristische Argumentationsmuster:  Der Artikel analysiert, wie Kläger Menschenrechte, Umweltschutzgesetze und staatliche Sorgfaltspflichten als Grundlage nutzen, um stärkere Maßnahmen gegen den Klimawandel zu erzwingen.
    \item Herausforderungen und Unterschiede: Es wird auf Unterschiede in den nationalen Rechtssystemen und die damit verbundenen Herausforderungen eingegangen, die den Erfolg von Klimaklagen maßgeblich beeinflussen.
    \item Ausblick auf die Entwicklung des Klimarechts: Abschließend werden mögliche zukünftige Trends und Entwicklungen im Bereich des globalen Klimarechts diskutiert, insbesondere im Hinblick auf die zunehmende Rolle von Gerichtsverfahren.
 \end{itemize}

 \textbf{\cite{businessportalnorwegenNorwegischeUmweltorganisationenGewinnen2024} Norwegische Umweltorganisationen gewinnen Klimaklage gegen den Staat}
 \begin{itemize}
    \item Klimaklage: Norwegische Umweltorganisationen haben eine erfolgreiche Klage gegen den norwegischen Staat geführt.
    \item Urteilsinhalt: Das Gerichtsurteil verpflichtet den Staat, strengere Klimaschutzmaßnahmen umzusetzen, um den Auswirkungen des Klimawandels besser entgegenzuwirken.
    \item Rechtliche Begründung: Die Entscheidung stützt sich auf die Pflicht des Staates, den Schutz der Umwelt und der Bürger sicherzustellen – insbesondere im Hinblick auf den Klimawandel und dessen Folgen.
    \item Bedeutung: Das Urteil wird als wegweisender Präzedenzfall angesehen, der den rechtlichen Rahmen für zukünftige Klimaschutzklagen stärkt.
    \item Implikationen: Neben der innerstaatlichen Wirkung könnten die Entscheidung auch internationale Diskussionen und Maßnahmen im Bereich des Klimaschutzes beflügeln.
 \end{itemize}

 \textbf{\cite{de2016shell} The ‘{{Shell}} Nigeria Issue’: Judgments by the Court of Appeal of {{The Hague}}, the {{Netherlands}}}

 \begin{itemize}
    \item Fallübersicht: Der Artikel befasst sich mit den Urteilen des Berufungsgerichts in Den Haag zum „Shell Nigeria Issue“, also zu Rechtsstreitigkeiten im Zusammenhang mit Shells Aktivitäten in Nigeria.
    \item Juristische Fragestellungen: Es wird untersucht, wie europäisches Gesellschaftsrecht – insbesondere Prinzipien der Corporate Governance und Haftung – auf die komplexen internationalen Strukturen von Shell angewendet werden kann.
    \item Analyse der Gerichtsentscheidungen: Der Artikel beleuchtet, inwiefern das Gericht die Verantwortlichkeit von Shell für mögliche Umwelt- und Menschenrechtsverletzungen in Nigeria beurteilt hat, und welche Rolle dabei die Abgrenzung zwischen Mutter- und Tochtergesellschaften spielte.
    \item Bedeutung für die Unternehmenspraxis:Es wird erörtert, welche Konsequenzen diese Urteile für multinationale Konzerne haben und wie sie als Präzedenzfälle für künftige Fälle transnationaler Unternehmenshaftung wirken können.
    \item Implikationen für das europäische Gesellschaftsrecht: Abschließend diskutiert der Beitrag, inwiefern diese Entscheidungen die Weiterentwicklung des europäischen Gesellschaftsrechts beeinflussen und welche Herausforderungen bei der Durchsetzung von Ansprüchen im internationalen Kontext bestehen.
 \end{itemize}

 \textbf{\cite{deutschlandfunk.deKlimaschutzBerufungsgerichtEntscheidet2024} Klimaschutz - Berufungsgericht entscheidet im Prozess gegen Shell}

 \begin{itemize}
    \item Wegweisendes Urteil: Ein Berufungsgericht hat im Rahmen eines Klimaschutzprozesses gegen Shell ein wegweisendes Urteil gefällt.
    \item Kernentscheidung: Das Gericht kam zu dem Schluss, dass Shell stärker in die Pflicht genommen werden muss, um wirksam zum Klimaschutz beizutragen – etwa durch die Reduktion von Treibhausgasemissionen.
    \item Rechtliche Begründung: Die Entscheidung stützt sich auf die Bewertung von Shells Beitrag zum Klimawandel und dessen Auswirkungen, was das Unternehmen in die Verantwortung nimmt.
    \item Folgen für Shell und den Klimaschutz: Das Urteil könnte weitreichende Konsequenzen für die Unternehmenspraxis von Shell haben und als Präzedenzfall für zukünftige Klimaklagen dienen.
    \item Reaktionen und Diskussion: Der Artikel beleuchtet zudem, wie verschiedene Akteure – von Umweltschützern bis hin zu politischen Entscheidungsträgern – auf das Urteil reagieren und welche Auswirkungen dies auf die weitere Klimapolitik haben könnte.
 \end{itemize}

 \textbf{\cite{deutschlandfunk.deKommentarShellGericht2024} Kommentar zu Shell: Das Gericht ist nicht eingeknickt}

 \begin{itemize}
    \item Standfestigkeit der Justiz: Der Kommentar betont, dass das Gericht trotz erheblicher politischer und wirtschaftlicher Erwartungen nicht \glqq{}eingeknickt\grqq{} ist, sondern eine eigenständige, sachliche Rechtsbewertung vorgenommen hat.
    \item Fundierte juristische Argumentation: Es wird hervorgehoben, dass das Urteil auf einer soliden juristischen Grundlage beruht, auch wenn es nicht alle Forderungen der Klimabewegung vollständig erfüllt.
    \item Kritik an Überinterpretationen: Der Beitrag kritisiert, dass manche Akteure das Urteil als zu schwach deuten oder als Zeichen einer Kapitulation der Justiz interpretieren – dabei wird die Unabhängigkeit des Gerichts verteidigt.
    \item Bedeutung für Klimaklagen: Das Urteil wird als wegweisender Präzedenzfall dargestellt, der zukünftige Klimaschutzprozesse und die Verantwortung von Unternehmen stärken könnte.
    \item Vertrauenssignal an den Rechtsstaat: Abschließend wird betont, dass das entschiedene und unabhängige Vorgehen des Gerichts wichtig für das Vertrauen in den Rechtsstaat und die Auseinandersetzung mit dem Klimawandel ist.
 \end{itemize}

 \textbf{\cite{deutschlandfunk.deShellVerliertKlimaProzess2021} Shell verliert Klima-Prozess - Das Urteil von Den Haag und die Folgen}

 \begin{itemize}
    \item Urteilsfundament: Das Gericht in Den Haag hat in einem Klimaprozess gegen Shell entschieden – Shell muss sich vor Gericht verantworten.
    \item Kern des Prozesses: Es ging um die Frage, inwieweit Shell für ihren Beitrag zum Klimawandel haftbar gemacht werden kann und ob das Unternehmen ausreichende Maßnahmen zur Emissionsreduktion ergriffen hat.
    \item Bedeutung des Urteils: Das Urteil unterstreicht die zunehmende Verantwortung von Großunternehmen im Kontext des Klimaschutzes und könnte als Präzedenzfall für zukünftige Klimaklagen dienen.
    \item Auswirkungen auf Shell: Die Entscheidung hat weitreichende Konsequenzen für das Geschäftsmodell und die zukünftige Strategie von Shell im Hinblick auf Umweltschutz und Nachhaltigkeit.
    \item Reaktionen und Folgen: Der Artikel beleuchtet auch die Reaktionen verschiedener Akteure – von Klimaaktivisten bis zu politischen Entscheidungsträgern – und diskutiert, welche langfristigen Folgen das Urteil für die Klimapolitik haben könnte.
 \end{itemize}

 \textbf{\cite{domans2021dutch} Dutch Court Order Shell to Reduce Emissions in First Climate Change Ruling against Company}
 
 \begin{itemize}
    \item Erstes Klimagerichtsurteil: Ein niederländisches Gericht hat erstmals ein Urteil gegen ein Unternehmen im Zusammenhang mit dem Klimawandel gefällt.
    \item Kernentscheidung: Shell wurde vom Gericht dazu verpflichtet, ihre Emissionen zu reduzieren, da ihre bisherigen Maßnahmen als unzureichend bewertet wurden.
    \item Juristischer Präzedenzfall: Das Urteil wird als bedeutender Präzedenzfall angesehen, der weitere rechtliche Schritte gegen Unternehmen im Kontext des Klimawandels beeinflussen könnte.
    \item Auswirkungen auf Shell: Die Entscheidung zwingt Shell, konkrete und messbare Maßnahmen zur Reduktion von Treibhausgasemissionen umzusetzen.
    \item Breitere Bedeutung: Das Urteil unterstreicht die wachsende rechtliche Verantwortung von Unternehmen hinsichtlich ihres Beitrags zum Klimawandel.
 \end{itemize}

 \textbf{\cite{Frank2019} The Case of Huaraz: {{First}} Climate Lawsuit on Loss and Damage against an Energy Company before German Courts}

 \begin{itemize}
   \item Fallbeschreibung: Der Artikel behandelt den Fall von Huaraz, bei dem eine Klimaklage gegen ein Energieunternehmen vor deutschen Gerichten geführt wurde. Es handelt sich um die erste Klage, die sich mit Verlust und Schaden (Loss and Damage) im Kontext des Klimawandels gegen ein Unternehmen richtet.
   \item Klagegrund: Kläger ist ein peruanischer Landwirt aus Huaraz, der argumentiert, dass das Unternehmen für die durch den Klimawandel verursachten Gefahren verantwortlich ist, wie die Bedrohung durch Gletscherschmelze und Überschwemmungen, die den Palcacocha-See in Huaraz betreffen.
   \item Rechtliche Grundlage: Der Fall ist bemerkenswert, da er auf die Haftung von Unternehmen für Schäden hinweist, die durch den Klimawandel verursacht werden, und auf die Rolle der Gerichte bei der Bestimmung von Verantwortlichkeiten für Klimarisiken.
   \item Ziel der Klage: Die Klage fordert das Unternehmen zu finanziellen Entschädigungen und Schutzmaßnahmen auf, die den Verlust und Schaden, der durch den Klimawandel verursacht wird, abmildern sollen.
   \item Bedeutung des Falls: Der Artikel diskutiert den breiteren Kontext des Verlusts und Schadens im Klimawandel und seine Relevanz für die rechtlichen, politischen und gesellschaftlichen Diskussionen rund um den Klimaschutz und die Unternehmensverantwortung.
 \end{itemize}

 \textbf{\cite{Hansen2023Destruktive} Destruktive ambiguität bremst fortschritte im UN-klimaprozess: In bonn standen zentrale säulen des pariser abkommens unter beschuss}

 \begin{itemize}
   \item Hauptthema des Artikels: Der Artikel von Gerrit Hansen untersucht, wie destruktive Ambiguität die Fortschritte im UN-Klimaprozess bremst, insbesondere bei den Verhandlungen, die 2023 in Bonn stattfanden.
   \item Zentrale Themen der Verhandlungen: Es wird darauf hingewiesen, dass bei den Verhandlungen wichtige Elemente des Pariser Abkommens unter Druck standen, insbesondere in Bezug auf Finanzierung, Emissionsziele und die Verantwortung der Industrieländer.
   \item Destruktive Ambiguität: Der Begriff \glqq{}destruktive Ambiguität\grqq{} wird verwendet, um die Unsicherheit und die widersprüchlichen Positionen innerhalb der Verhandlungen zu beschreiben, die den Fortschritt im Klimaschutz behindern.
   \item Konflikte bei den Verhandlungen: Die Verhandlungen in Bonn wurden von Spannungen zwischen verschiedenen Ländern und Interessengruppen dominiert, die unterschiedliche Vorstellungen über die Umsetzung und Weiterentwicklung des Pariser Abkommens haben.
   \item Forderung nach klareren Zielen: Hansen plädiert dafür, dass die UN-Klimaverhandlungen klarere, verbindlichere Ziele und Verpflichtungen festlegen müssen, um tatsächliche Fortschritte im Klimaschutz zu erzielen.
   \item Bedeutung für den globalen Klimaschutz: Der Artikel schließt mit der Feststellung, dass die zukünftige Gestaltung des UN-Klimaprozesses von entscheidender Bedeutung ist, um die globalen Klimaziele zu erreichen und eine nachhaltige, gerechte Klimapolitik zu entwickeln.
 \end{itemize}

 \textbf{\cite{hesselman2024milieudefensie} Milieudefensie v Shell: 3 Takeaways and Challenges on the Appeal’s Court Decision}

 \begin{itemize}
   \item Überblick über das Urteil: Der Artikel beleuchtet die Entscheidung im Fall Milieudefensie v Shell und fasst die drei zentralen Erkenntnisse (Takeaways) sowie die damit verbundenen Herausforderungen zusammen.
   \item Erstes Takeaway – Stärkung der Unternehmensverantwortung: Das Urteil unterstreicht, dass Unternehmen wie Shell stärker in die Pflicht genommen werden müssen, um ihre Emissionen zu reduzieren und ihrer Verantwortung im Klimaschutz gerecht zu werden.
   \item Zweites Takeaway – Fortschritte in der Klimarechtsprechung: Es wird hervorgehoben, dass das Urteil als Meilenstein für die Klimarechtsprechung gilt, indem es neue Maßstäbe für die Haftung von Unternehmen im Kontext des Klimawandels setzt. Gleichzeitig werden bestehende juristische Hürden, wie etwa Fragen zur Beweisführung, diskutiert.
   \item Drittes Takeaway – Ausblick und Präzedenzfall für künftige Klagen: Das Urteil bietet wichtige Anhaltspunkte dafür, wie zukünftige Klimaklagen gestaltet werden können und welche rechtlichen Entwicklungen im internationalen Vergleich zu erwarten sind.
   \item Zentrale Herausforderungen: Neben den positiven Impulsen zeigt der Artikel auch auf, dass es noch ungelöste Herausforderungen gibt – etwa hinsichtlich der Durchsetzbarkeit des Urteils und der internationalen Harmonisierung von Klimaschutzstandards.
 \end{itemize}
 
\textbf{\cite{Kanning_2024} Shell and the Climate Case: {{Is}} the Shell Group the “Cheapest Cost Avoider”?}

\begin{itemize}
   \item Hauptthema des Artikels: Der Artikel untersucht, ob die Shell Group als der „günstigste Kostenvermeider“ im Klimafall betrachtet werden kann und welche rechtlichen sowie wirtschaftlichen Implikationen dies hat.
   \item Definition des Begriffs „Cheapest Cost Avoider“: Der Begriff bezieht sich auf die Idee, dass das Unternehmen, das am günstigsten in der Lage ist, die gesellschaftlichen Kosten zu vermeiden (z.B. durch Reduzierung von Emissionen), auch die Verantwortung tragen sollte, diese Kosten zu vermeiden.
   \item Shells Verantwortung im Klimafall: Kanning analysiert, inwieweit Shell als Unternehmen tatsächlich die Verantwortung trägt, den Klimawandel durch angemessene Maßnahmen zu bekämpfen, und ob es als der „günstigste Kostenvermeider“ betrachtet werden kann.
   \item Rechtliche Perspektive: Der Artikel diskutiert die rechtlichen Argumente, die von den Klägern im Klimafall gegen Shell verwendet werden, und wie die Justiz diese zur Beurteilung der Verantwortung des Unternehmens heranzieht.
   \item Ökonomische Implikationen: Die ökonomischen Auswirkungen einer solchen Betrachtung werden ebenfalls erörtert, insbesondere die Kosten für Shell und die gesellschaftlichen Vorteile einer schnelleren Emissionsreduktion.
   \item Ausblick und Bedeutung des Falls: Kanning schließt mit einer Diskussion darüber, wie der Fall und die Argumentation des „günstigsten Kostenvermeiders“ zukünftige Klimaklagen und die Verantwortlichkeit von Unternehmen im internationalen Recht beeinflussen könnten.
\end{itemize}

\textbf{\cite{KlimaklageUrgendaGegen} Klimaklage Urgenda gegen Niederlande}

\begin{itemize}
   \item Hintergrund der Klage: Im Jahr 2013 reichte die niederländische Umweltorganisation Urgenda gemeinsam mit 886 Bürgern eine Klage gegen den niederländischen Staat ein. Sie forderten eine stärkere Reduzierung der Treibhausgasemissionen, um die Bevölkerung vor den Folgen des Klimawandels zu schützen.
   \item Erstinstanzliches Urteil (2015): Das Gericht in Den Haag entschied 2015 zugunsten von Urgenda und verpflichtete den Staat, die CO$_2$-Emissionen bis 2020 um mindestens 25 \% gegenüber dem Niveau von 1990 zu senken.
   \item Berufungsurteil (2018): Im Oktober 2018 bestätigte das Berufungsgericht das Urteil und betonte die Pflicht des Staates zum Schutz des Lebens und Wohlbefindens seiner Bürger gemäß der Europäischen Menschenrechtskonvention.
   \item Letztinstanzliches Urteil (2019): Der Hohe Rat der Niederlande wies im Dezember 2019 die Revision des Staates zurück und bestätigte die Verpflichtung zur Emissionsreduktion. Dieses Urteil gilt als historisch und ist das erste erfolgreiche Verfahren, in dem ein Staat zur Einhaltung konkreter Klimaziele verurteilt wurde.
   \item Auswirkungen des Urteils: Das Urteil führte zu einer Verschärfung der Klimapolitik in den Niederlanden und setzte einen Präzedenzfall für ähnliche Klagen weltweit.
\end{itemize}

\textbf{\cite{konneke2024deutsch} Die Deutsch-Brasilianische Partnerschaft Für Sozial-Ökologische Transformation: {{Bilaterale}} Zusammenarbeit Als Katalysator Für Den {{UN-klimaprozess}}}

\begin{itemize}
   \item Hauptthema des Artikels: Der Artikel von Jule Könneke analysiert die Rolle der deutsch-brasilianischen Partnerschaft für sozial-ökologische Transformation und wie diese bilaterale Zusammenarbeit als Katalysator für den UN-Klimaprozess dienen kann.
   \item Herausforderungen der globalen Klimakooperation: Es wird betont, dass die zunehmenden Spannungen zwischen dem Globalen Norden und Süden die weltweite Klimakooperation erschweren und Deutschlands Suche nach verlässlichen Partnern beeinflussen
   \item Brasiliens Rolle als Brückenland: Brasilien wird als Schlüsselakteur hervorgehoben, der dazu beitragen kann, Spannungen abzubauen, insbesondere als Gastgeber der Weltklimakonferenz 2025 (COP30). 
   \item Ziele der Partnerschaft: Deutschland und Brasilien sollten im Rahmen ihrer Partnerschaft darauf hinarbeiten, das Vertrauen in die Klimaverhandlungen zu stärken und eine effektive Zusammenarbeit zwischen handlungsfähigen und -willigen Regierungen aus Nord und Süd zu fördern.
   \item Aktuelle globale Rahmenbedingungen: Der Artikel beschreibt den zunehmend multipolaren Kontext, in dem der Zugang zu Ressourcen und Märkten für grüne Technologien umkämpft ist, sowie die geopolitischen Machtverschiebungen und die fortschreitende Klimakrise.
   \item Empfehlungen für die Partnerschaft: Es wird empfohlen, dass Deutschland und Brasilien gemeinsam Initiativen ergreifen, um das Vertrauen in den UN-Klimaprozess zu stärken und als Vorreiter für eine sozial-ökologische Transformation zu agieren.
\end{itemize}

\textbf{\cite{ltoHistorischesKlimaurteilGegen2024} Historisches Klimaurteil gegen Shell kassiert}

\begin{itemize}
   \item Aufhebung des historischen Klimaurteils: Ein Berufungsgericht in Den Haag hat das zuvor ergangene Urteil von 2021, das Shell verpflichtete, seine CO$_2$-Emissionen bis 2030 um 45 \% zu reduzieren, aufgehoben.
   \item Begründung des Gerichts: Das Gericht betonte zwar Shells Verantwortung im globalen Klimaschutz, setzte jedoch keine spezifischen Reduktionsziele fest. Es argumentierte, dass eine solche Verpflichtung für ein einzelnes Unternehmen zu negativen globalen Klimafolgen führen könnte, wie beispielsweise einer erhöhten Kohleförderung in anderen Regionen. 
   \item Reaktionen auf das Urteil: Während Shell das Urteil als Sieg betrachtet, sehen Umweltschützer darin einen Rückschlag für den Klimaschutz. Sie befürchten, dass dieses Urteil als Präzedenzfall dienen könnte, um die Verantwortung von Unternehmen im Klimaschutz zu mindern.
   \item Mögliche weitere Schritte: Es wird erwartet, dass die Kläger, insbesondere die Umweltorganisation Milieudefensie, Revision beim höchsten niederländischen Gericht, dem Hohen Rat, einlegen werden.
\end{itemize}

\textbf{\cite{Mayer_2022} The Duty of Care of Fossil-Fuel Producers for Climate Change Mitigation: {{Milieudefensie}} v. Royal Dutch Shell District Court of the Hague (the Netherlands)}

\begin{itemize}
   \item Zentrale Thematik: Der Artikel untersucht die Sorgfaltspflicht von fossilen Brennstoffproduzenten bei der Minderung des Klimawandels und beleuchtet dabei insbesondere das Verfahren Milieudefensie v. Royal Dutch Shell am Bezirksgericht Den Haag.
   \item Rechtliche Argumentation: Es wird diskutiert, ob und inwieweit fossile Brennstoffunternehmen eine rechtliche Verpflichtung (Duty of Care) besitzen, um wirksame Maßnahmen zur Reduzierung von Treibhausgasemissionen zu ergreifen und somit ihren Beitrag zur Klimakrise zu minimieren.
   \item Fallanalyse: Der Artikel analysiert das konkrete Verfahren gegen Shell, bei dem die Kläger argumentieren, dass das Unternehmen durch seine Emissionen einen erheblichen Beitrag zum Klimawandel leistet und daher zur Einhaltung strengerer Emissionsziele verpflichtet ist.
   \item Juristische Grundlagen und Präzedenzfälle: Es werden die relevanten rechtlichen Grundlagen sowie frühere Gerichtsurteile herangezogen, um die Argumentation der Sorgfaltspflicht zu untermauern. Dabei spielt auch die Verknüpfung von Umweltschutz und Menschenrechten eine zentrale Rolle.
   \item Implikationen für die Zukunft: Der Artikel zeigt auf, dass das Urteil weitreichende Konsequenzen für die Haftung fossiler Brennstoffproduzenten haben könnte und als Präzedenzfall für künftige Klimaklagen weltweit dient.
\end{itemize}

\textbf{\cite{RechtlichesClimateCase2022} Rechtliches | The climate case - Saúl vs. RWE}

\begin{itemize}
   \item Hintergrund des Falls: Der peruanische Bauer und Bergführer Saúl Luciano Lliuya aus Huaraz klagt gegen den deutschen Energiekonzern RWE. Er macht das Unternehmen für die Auswirkungen des Klimawandels verantwortlich, die sein Eigentum und seine Heimatstadt bedrohen.
   \item Klagegrundlage: Luciano Lliuya argumentiert, dass RWE als einer der größten CO$_2$-Emittenten der Welt mitverantwortlich für das Abschmelzen der Gletscher in den Anden ist. Dies führt zu einem Anstieg des Wasserspiegels des Palcacocha-Sees, wodurch die Stadt Huaraz und seine Liegenschaften einem erhöhten Überschwemmungsrisiko ausgesetzt sind.
   \item Forderungen des Klägers: Er verlangt von RWE einen finanziellen Beitrag zu den Schutzmaßnahmen gegen die drohende Flutkatastrophe, entsprechend dem Anteil des Unternehmens an den globalen CO$_2$-Emissionen, der laut Kläger bei etwa 0,47 \% liegt. 
   \item Verfahrensverlauf: Die Klage wurde 2015 beim Landgericht Essen eingereicht und zunächst abgewiesen. In der Berufung entschied das Oberlandesgericht Hamm 2017, in die Beweisaufnahme einzutreten. 
   \item Aktueller Stand: Im Mai 2022 reisten deutsche Richter, Anwälte und Sachverständige nach Huaraz, um die örtlichen Gegebenheiten zu begutachten und Beweise zu sammeln. Sie entnahmen Wasserproben aus dem Palcacocha-See und führten Drohnenaufnahmen durch. Diese Untersuchung soll klären, ob tatsächlich ein erhöhtes Risiko für Überschwemmungen besteht, das durch den Klimawandel und somit durch die CO$_2$-Emissionen von Unternehmen wie RWE verursacht wird. 
\end{itemize}

\textbf{\cite{sato2024impacts} Impacts of Climate Litigation on Firm Value}

\begin{itemize}
   \item Hauptthema des Artikels: Die Studie untersucht, wie sich Klimaklagen auf den Unternehmenswert auswirken.
   \item Durchschnittlicher Einfluss auf Aktienrenditen: Unternehmen verzeichnen im Durchschnitt einen Rückgang der Aktienrenditen um 0,41 \% nach einer klimabezogenen Klageeinreichung oder einer ungünstigen Gerichtsentscheidung.
   \item Variabilität der Auswirkungen: Die Auswirkungen variieren je nach Unternehmensgröße, Branche und Art der Klage.
   \item Langfristige Effekte: Neben kurzfristigen Marktreaktionen können solche Klagen langfristige Auswirkungen auf den Unternehmenswert haben, insbesondere wenn sie zu regulatorischen Änderungen oder Reputationsverlusten führen.
   \item Implikationen für Investoren: Investoren sollten Klimarisiken und potenzielle Rechtsstreitigkeiten in ihre Bewertungsmodelle integrieren, um fundierte Entscheidungen treffen zu können.
\end{itemize}

\textbf{\cite{ShellBegruesstUrteil2024} Shell begruesst Urteil des niederlaendischen Berufungsgerichts | Über uns: Shell in Deutschland}

\begin{itemize}
   \item Offizielle Stellungnahme: Shell begrüßt das Urteil des niederländischen Berufungsgerichts ausdrücklich und bezieht sich damit auf die jüngsten Entscheidungen im Zusammenhang mit klimabezogenen Rechtsstreitigkeiten.
   \item Rechtliche Bestätigung: Das Urteil wird als Bestätigung der Rechtssicherheit in Bezug auf Shells Klima- und Energiepolitik dargestellt. Es unterstreicht, dass die getroffenen Maßnahmen und Strategien den geltenden gesetzlichen Rahmenbedingungen entsprechen.
   \item Bedeutung für die Unternehmensstrategie: Shell betont, dass das Urteil ihre laufenden Bemühungen im Bereich Klimaschutz und Dekarbonisierung unterstützt und als Ansporn dient, die Energiewende weiter voranzutreiben.
   \item Zukunftsausrichtung: Das Unternehmen bekräftigt sein Engagement, kontinuierlich in nachhaltige Technologien zu investieren und mit allen relevanten Akteuren zusammenzuarbeiten, um die Herausforderungen des Klimawandels zu meistern.
\end{itemize}

\textbf{\cite{ParisAgreementUNFCCC} The {{Paris Agreement}} | {{UNFCCC}}}

\begin{itemize}
   \item Globales Klimaziel: Begrenzung des globalen Temperaturanstiegs auf deutlich unter 2 °C, mit dem Bestreben, 1,5 °C nicht zu überschreiten.
   \item Nationale Beiträge (NDCs): Verpflichtung für alle Vertragsparteien, regelmäßig eigene Klimaschutzpläne zu erstellen, umzusetzen und zu aktualisieren.
   \item Langfristige Zielsetzung: Erreichen von Klimaneutralität in der zweiten Hälfte des 21. Jahrhunderts, um die langfristigen Folgen des Klimawandels einzudämmen.
   \item Unterstützung für Entwicklungsländer: Förderung finanzieller, technischer und Kapazitätsaufbau-Maßnahmen zur Unterstützung von Entwicklungsländern bei der Anpassung an den Klimawandel und bei der Emissionsminderung.
   \item Transparenz und Überprüfung: Etablierung von Mechanismen zur regelmäßigen Überwachung, Berichterstattung und Überprüfung der Fortschritte bei der Umsetzung der nationalen Beiträge.
\end{itemize}

\textbf{\cite{Vanhala01052013} The Comparative Politics of Courts and Climate Change}

\begin{itemize}
   \item Untersuchungsfokus: Analyse der Rolle von Gerichten im Umgang mit Klimawandel und deren Einfluss auf die politische Entscheidungsfindung.
   \item Vergleichende Perspektive: Betrachtung unterschiedlicher Rechtssysteme (z. B. Common Law vs. Civil Law) und wie diese die gerichtliche Intervention in Klimafragen prägen.
   \item Gerichtliche Einflussnahme: Diskussion, inwiefern Gerichte als Katalysatoren für Klimapolitik wirken können – sowohl als Förderer ambitionierter Maßnahmen als auch bei der Beibehaltung staatlicher Untätigkeit.
   \item Determinanten des gerichtlichen Handelns: Untersuchung, welche Faktoren (politische Institutionen, rechtliche Traditionen, öffentliche Meinung) das Verhalten und die Entscheidungsfindung von Gerichten in Klimafragen beeinflussen.
   \item Auswirkungen auf die Umweltpolitik: Bewertung, wie gerichtliche Entscheidungen langfristig die Gestaltung und Umsetzung von Klimapolitiken sowie die Entwicklung des Umweltrechts beeinflussen.
   \item Methodischer Ansatz: Einsatz eines vergleichenden politikwissenschaftlichen Rahmens, um die Wechselwirkungen zwischen Gerichten und Klimapolitik in verschiedenen nationalen Kontexten zu beleuchten.
\end{itemize}

\clearpage

%---------------------------------------------------------
% Bibliografie
%---------------------------------------------------------
\begingroup
\renewcommand{\bibfont}{\fontsize{13pt}{12pt}\selectfont}  
\sloppy
\nocite{*}
\printbibliography{}

\end{document}
